\documentclass[11pt,a4paper,oneside]{book}

\usepackage{fancyhdr}
\usepackage{verbatim} 
\usepackage{hyperref}

\setlength{\headheight}{15.2pt}
\pagestyle{fancy}

\begin{comment}
\fancyhf{}
 \lhead{CS245 Course Reader}
\rhead{Abi Raja}
\rfoot{\thepage}
\end{comment}

\begin{document}

\title{ { \huge \bfseries The Book}}
\date{Febuary 2011}
\maketitle

\tableofcontents
 \clearpage
     
\chapter{Preface}

I'm a big believer in cheatsheet learning. There are two kinds of knowledge in this world, stuff that you need to memorize and stuff that you need to understand. Names need to be memorized, concepts need to be understood. But you really need both names and concepts. In this book, we'll have a bunch of cheatsheets and we'll explain the concepts in great detail so you can BEAST everything related to JavaScript.

From the Git Parable by Tom:

"Git is a simple, but extremely powerful system. Most people try to teach Git by demonstrating a few dozen commands and then yelling �tadaaaaa.� I believe this method is flawed. Such a treatment may leave you with the ability to use Git to perform simple tasks, but the Git commands will still feel like magical incantations. Doing anything out of the ordinary will be terrifying. Until you understand the concepts upon which Git is built, you�ll feel like a stranger in a foreign land."

\clearpage

\section*{Acknowledgements}

God.

\chapter{Operating Systems}


\chapter{CoffeeScript}

CoffeeScript sucks. Never use it. 

Or, do use it, but before you use it, write a CoffeeScript debugger first.

\chapter{Compilers}


\chapter{Reading List}

A collection of chapters from books or webpages that are worth reading as supplementary material:

\begin{itemize}
	\item MySQL Manual: In particular
\end{itemize}

\end{document}
