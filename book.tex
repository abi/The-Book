\documentclass[11pt,a4paper,oneside]{book}

\usepackage{fancyhdr}
\usepackage{listings}
\usepackage{verbatim}
\usepackage{hyperref}
\usepackage{color}
\usepackage{xcolor}
\usepackage{caption}

\DeclareCaptionFont{white}{\color{white}}
\DeclareCaptionFormat{listing}{\colorbox{gray}{\parbox{\textwidth}{#1#2#3}}}
\captionsetup[lstlisting]{format=listing,labelfont=white,textfont=white}

\setlength{\headheight}{15.2pt}
\pagestyle{fancy}

\begin{comment}
\fancyhf{}
 \lhead{CS245 Course Reader}
\rhead{Abi Raja}
\rfoot{\thepage}
\end{comment}

\begin{document}

\title{ { \huge \bfseries The Book}}
\date{Febuary 2011}
\maketitle

\tableofcontents
 \clearpage
     
\chapter{Preface}

I'm a big believer in cheatsheet learning. There are two kinds of knowledge in this world, stuff that you need to memorize and stuff that you need to understand. Names need to be memorized, concepts need to be understood. But you really need both names and concepts. In this book, we'll have a bunch of cheatsheets and we'll explain the concepts in great detail so you can BEAST everything related to JavaScript.

From the Git Parable by Tom:

"Git is a simple, but extremely powerful system. Most people try to teach Git by demonstrating a few dozen commands and then yelling �tadaaaaa.� I believe this method is flawed. Such a treatment may leave you with the ability to use Git to perform simple tasks, but the Git commands will still feel like magical incantations. Doing anything out of the ordinary will be terrifying. Until you understand the concepts upon which Git is built, you�ll feel like a stranger in a foreign land."

\clearpage

\section*{Acknowledgements}

God.

\chapter{Operating Systems}


\chapter{CoffeeScript}

CoffeeScript sucks. Never use it.

First: What's so bad about Javascript that you need to cover it up with an intermediary layer, like an ugly child?

\begin{itemize}

\item Prototypal inheritance? Just learn it. Or don't use it.

\item \verb!==/===! confusion? \verb!for (var i in something)! breakage? JS is not a perfect language. It was only written in 10 days. Is that enough to shun a language? Not really.

\item Curly brackets? Arguable, if you're really about language aesthetics (I'm not). If you really don't like curly brackets, then just use CoffeeScript, but don't actually write any CoffeeScript code except for indentation.

\end{itemize}

Javascript is a beautiful language (with maybe a few ugly corners) - but that's an essay for another time.

The problem with CoffeeScript is simple and inevitable: Cross referencing generated code to Javascript code to CS is really difficult. Debugging is much slower than debugging straight JS.

\chapter{Easy Javascript errors}

%Should it be JavaScript or Javascript?

There are a lot of subtle ways to make mistakes in Javascript.

\section{The in operator}

\begin{lstlisting}[label=forin1,caption=Looks like python.]
if (5 in [1,2,3]) {
  //...do stuff
}
\end{lstlisting}

The 'in' operator doesn't work on arrays in the way you'd expect.

What's actually going on ? When you ask \verb!x in array!, Javascript considers the array as an object that maps indicies to values, so \verb! 0 in [5,1]! is similar to asking \verb! 0 in {0 : 5, 1 : 1}!. Cool, unless you get tricked by \verb!0 in [0]!.

A better solution to this problem:

\begin{lstlisting}[label=forin2,caption=A solution]
[1,2,3].indexOf(5) !== -1
\end{lstlisting}

\subsection{Another trick}

Watch out for stuff like:

\begin{lstlisting}[label=forin3,caption=Same bug.]
for (x in [1,2,3,4]){
  //...stuff
}
\end{lstlisting}

This fails for the same reason.

\section{parseInt}

\begin{lstlisting}[label=parseint,caption=That 0 looks innocuous enough...]
console.log(parseInt('08'));
\end{lstlisting}

Go ahead, try it out. I'll be waiting.

It prints 0! The 0 in front of the 8 indicates that the number is to be parsed as octal. Since 8 is not a valid octal number, it defaults to 0.
The correct way to use parseInt is to specify the radix as the second parameter. This will always work:

\begin{lstlisting}[label=parseint2,caption=Fixed.]
console.log(parseInt('08', 10));
\end{lstlisting}


\chapter{Compilers}


\chapter{Reading List}

A collection of chapters from books or webpages that are worth reading as supplementary material:

\begin{itemize}
	\item MySQL Manual: In particular
\end{itemize}

\end{document}
